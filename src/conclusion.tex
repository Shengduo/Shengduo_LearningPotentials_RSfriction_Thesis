\section{Conclusions}
\label{sec:conclusions}
\ifincludetext{
In this study, 
we have developed a potential-formulated friction that resembles the widely-used rate-and-state friction law. 
By constructing the potentials using Recurrent Neural Operators (RNOs) and training them on generated rate-and-state slip-friction sequences, 
we have verified that the potential friction formulation can capture the history dependencies well in rate-and-state friction, 
which is widely observed from experimental data. 
This suggests that there exists a potential-formulated friction law of our specific form that approximates the empirical rate-and-state friction well, 
with a relative $L_2$ error in friction coefficient of $2 \times 10^{-4}$. 
This is 2 orders of magnitude lower than the error between the best rate-and-state friction fit to real experimental data, 
which implies that they likely will fit the experimental data similarly well. 
We testify through different training runs that in our potential formulation, hidden variable $\xi$ is unique up to affine transformations.

We have also confirmed by solving spring-slider systems that the potential friction formulation can facilitate implicit solves of initial value problems with rate-and-state frictional interfaces, 
since the propagation of solution to the next time step can be written as a convex minimization problem. 
We test whether or not the proposed potential formulation facilitates implicit solves of initial value problems by solving spring-slider dynamic problems under displacement-control loading. 
We sample $77$ sequences with different spring constants and loading, 
and out of them rate-and-state friction cannot solve over $50\%$ of the sequences with implicit solver, $46\%$ of the sequences with explicit solver. 
While potential-formulated friction can solve all of these sequences, 
with either implicit or explicit solver. 
For the explicit solver, 
rate-and-state friction still cannot solve those sequences with a time step $1/256$ of that of the potential-formulated friction.  ($2^{-19}$ vs $2^{-11}\ \mathrm{s}$). 
Within the sequences that both explicit rate-and-state and implicit potential friction can solve, 
they achieve similar accuracy as time step increases. 

With all the advantages of the potential formulation for rate-and-state friction, 
there is one drawback worth mentioning. 
The potentials require a large number of sequences in the training dataset to achieve good test error. 
Since compared to the original rate-and-state friction law, 
there are much more parameters in the potentials from their Neural Network structure, 
hundreds of sequences are required to avoid over-fitting and achieve good test error. 
In practice, 
it is sometimes difficult to obtain hundreds of sequences from experiments done on the same frictional interface, 
while the original rate-and-state friction law in general requires a handful of sequences to fit the four parameters. 

In the future, 
we would work on finding closed-form approximations to the learnt potentials, 
such that taking their gradients can be done more time-efficiently. 
Right now since the potentials are still neural networks, 
the differentiation process is time consuming. 
We would work on fitting the potential formulations directly to experimental measurements and compare the fitting error with the original rate-and-state formulation. 
}
\fi